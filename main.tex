% Ref: https://www.overleaf.com/latex/templates/emory-poster-template/skpfmpxjnqdh

\documentclass[25pt, a0paper, landscape, margin=10mm, innermargin=15mm, blockverticalspace=15mm, subcolspace=8mm, dvipsnames]{tikzposter} % you need to leave in dvipsnames or else it undoes the orange edge colour??

\usepackage[T1]{fontenc}
\usepackage{helvet}
%\usepackage[utf8]{inputenc}
\usepackage{amsmath}
\usepackage{amsfonts}
\usepackage{amsthm}
\usepackage{amssymb}
\usepackage{mathrsfs}
\usepackage{graphicx}
\usepackage{adjustbox}
\usepackage{enumitem}
\usepackage[backend=bibtex,style=numeric, citestyle=ieee]{biblatex}
\usepackage{xpatch}
\usepackage{xcolor}
\usepackage{multicol}
\usepackage{lipsum}
\usepackage{setspace}

\setlength{\columnsep}{2.5cm}
\setlength{\columnseprule}{1mm}
\pagecolor{Dandelion}

% Ref: https://latex-cookbook.net/poster/
\renewcommand*{\familydefault}{\sfdefault}% Let's have a sans serif font

% set theme parameters
\tikzposterlatexaffectionproofoff
\usepackage{anyfontsize}
\usetheme{Default}
\usebackgroundstyle{Default}
\definecolor{epccnavy}{HTML}{1D2A3D}
\definecolor{universityred}{HTML}{D50032}
\colorlet{backgroundcolor}{white} % <<< this makes bg white
\colorlet{framecolor}{epccnavy}%Dandelion}
\colorlet{titlefgcolor}{epccnavy}
\colorlet{blocktitlebgcolor}{epccnavy}
\colorlet{blocktitlefgcolor}{white}
\colorlet{blockframecolor}{epccnavy}

% \definetitlestyle{sampletitle}{
%     width=800mm, 
%     roundedcorners=20, 
%     linewidth=10pt, 
%     innersep=10pt,
%     titletotopverticalspace=15mm, titletoblockverticalspace=25mm 
% }{
% \begin{scope}[line width=\titlelinewidth, rounded corners=\titleroundedcorners]
% \draw[color=blocktitlebgcolor, fill=titlebgcolor]
% (\titleposleft,\titleposbottom) rectangle (\titleposright,\titlepostop);
% \end{scope}
% }
% \usetitlestyle[]{sampletitle}


% Ref: https://tex.stackexchange.com/questions/309713/modify-font-style-in-title-of-tikzposter
\settitle{ \centering
\vbox{
    \begin{spacing}{1.5}
    %\@titlegraphic \\[\TP@titlegraphictotitledistance] \centering
    \color{titlefgcolor} {\sffamily \bfseries \huge  \textsc{\@title} \par}
    \vspace*{0.75em}
    \color{universityred} {\sffamily \bfseries \huge \subtitle}   
    \vspace*{0.5em}
    \begin{flushleft}
    {\color{titlefgcolor} {\sffamily \huge \@author}
    \hspace{1em}
    {\sffamily \LARGE \@institute \par}}
    \end{flushleft}
    \end{spacing}
}
}
\makeatother

% Ref: https://tex.stackexchange.com/questions/180234/how-can-i-make-my-title-wrap-in-a-tikzposter
\makeatletter
\def\title#1{\gdef\@title{\scalebox{\TP@titletextscale}{%
			\begin{minipage}[t]{\linewidth}
				%\centering
				#1
			\par
				\vspace{0.5em}
			\end{minipage}%
}}}
\makeatother

% Ref: https://tex.stackexchange.com/questions/263563/add-logos-beyond-the-title-tikzposter
%\title{\parbox{\linewidth}{Do Research Software Developer Personas Exist? Are *YOU* an RS-10X? \\ Identifying Distinct Developer/Repository Interaction Types by Mining GitHub Data}}
\title{\parbox{\linewidth}{\fontseries{bx}\selectfont {\fontsize{78}{78}\selectfont {Identifying Research Software Developer Personas Could Create Better Research Software Developers.} }}}
\newcommand{\subtitle}{Are YOU an RS-10X? Mining GitHub to Describe Developer/Repository Interaction Types}
\author{\textbf{Felicity `Flic' Anderson\textsuperscript{$\dagger$}}, Julien Sindt\textsuperscript{$\dagger$} and Neil Chue Hong\textsuperscript{$\dagger$}}
\institute{\textsuperscript{$\dagger$}EPCC, University of Edinburgh}
%\titlegraphic{\includegraphics{epcclogo.png}}

\makeatletter
\newcommand\insertlogoi[2][]{\def\@insertlogoi{\includegraphics[#1]{#2}}}
%\newcommand\insertlogoi[2][]{\def\@insertlogoi{\hspace*{0.5in}\includegraphics[#1]{#2}}}
\newcommand\insertlogoii[2][]{\def\@insertlogoii{\includegraphics[#1]{#2}}}
%\newcommand\insertlogoii[2][]{\def\@insertlogoii{\hspace*{-7.5in}\includegraphics[#1]{#2}}}
%\newcommand\insertlogoiii[2][]{\def\@insertlogoiii{\includegraphics[#1]{#2}}}
\newlength\LogoSep
\setlength\LogoSep{0pt}

\insertlogoi[width=14cm]{informaticsUoE.png}
\insertlogoii[width=15cm]{epcclogo.png}
%\insertlogoii[width=15cm]{EpccANDEmailQRsidebyside.png}

\renewcommand\maketitle[1][width=800mm]{  % #1 keys
	\normalsize
	\setkeys{title}{#1}
	% Title dummy to get title height
	\node[transparent,inner sep=\TP@titleinnersep, line width=\TP@titlelinewidth, anchor=north, minimum width=\TP@visibletextwidth-2\TP@titleinnersep]
	(TP@title) at ($(0, 0.5\textheight-\TP@titletotopverticalspace)$) {\parbox{\TP@titlewidth-2\TP@titleinnersep}{\TP@maketitle}};
	\draw let \p1 = ($(TP@title.north)-(TP@title.south)$) in node {
		\setlength{\TP@titleheight}{\y1}
		\setlength{\titleheight}{\y1}
		\global\TP@titleheight=\TP@titleheight
		\global\titleheight=\titleheight
	};
	
	% Compute title position
	\setlength{\titleposleft}{-0.5\titlewidth}
	\setlength{\titleposright}{\titleposleft+\titlewidth}
	\setlength{\titlepostop}{0.5\textheight-\TP@titletotopverticalspace}
	\setlength{\titleposbottom}{\titlepostop-\titleheight}
	
	% Title style (background)
	\TP@titlestyle
	
	% Title node
	\node[inner sep=\TP@titleinnersep, line width=\TP@titlelinewidth, anchor=north, minimum width=\TP@visibletextwidth-2\TP@titleinnersep]
	at (0,0.5\textheight-\TP@titletotopverticalspace)
	(title)
	{\parbox{\TP@titlewidth-2\TP@titleinnersep}{\TP@maketitle}};
	
	\node[inner sep=0pt,anchor=west] 
	at ([xshift=-\LogoSep]title.west)
	{\@insertlogoi};
	
	\node[inner sep=0pt,anchor=east] 
	at ([xshift=\LogoSep]title.east)
	{\@insertlogoii};
	
	% Settings for blocks
	\normalsize
	\setlength{\TP@blocktop}{\titleposbottom-\TP@titletoblockverticalspace}
}
\makeatother

\onehalfspacing
% begin document
\begin{document}
%\useblockstyle{Basic}
\maketitle
\begin{columns}
\column{.5}

\block{0: Conclusions}{
    {\fontsize{45}{45}\selectfont 
    %\vspace{1em}
    \begin{itemize}
    \setlength{\itemindent}{0.5em}
    \setlength\itemsep{0.2em}
        \item \textbf{Mining Research Software (RS) repository data can help understand RS development practices.}  
        \item Assignment to and engagement with GitHub features Issue Tickets (ITx) and Pull Requests (PRs) correlate with commit contributions.
        \item These metrics have limitations, but in combination with other properties may give a fuller picture of development responsibilities or activity.
        \item \textbf{Can this approach locate distinct clusters of behaviours and define and describe RS developer personas? Can they predict effectiveness?} 
    \end{itemize}
    \vspace{1em}
    }
    {\fontsize{45}{45}\selectfont
    This pilot study attempts to identify `superstar developers' within 10 larger RS repositories by exploring assignment and contributions data.  
    %\vspace{1em}
    }        % via https://tex.stackexchange.com/a/499070
	} 
\block{1: Approach}{
    {\fontsize{40}{40}\selectfont 
        Research Software is any software used to generate research, across all academic fields. 
        % Dev Personas in OSS; SE  
        % What do we know about how RSEs interact with their repos?  Repo usage at all %?  
        % Data Mining background as a method 

        https://\textbf{zenodo}.org/api/records API was queried for software records with a DOI (common in published research software); 
        records were selected with a `github.com' URL. These were presumed to be RS repositories. 
        
        These repositories were mined via \textbf{GitHub REST API} to gather repository statistics (xxxxxxx, yyyyyyyy, zzzzzzzz), contributor data (xxxxxxx, yyyyyyyy, zzzzzzzz), and issue ticket data and pull request data (xxxxxxx, yyyyyyyy, zzzzzzzz).  

        % inclusion/exclusion criteria... languages, size, age, etc 

        % table of stats for total repo dataset N=1823 number of devs; average commits; average repo age; average number of issue tickets / PRs; 


        This larger dataset was subset to select repositories with larger numbers of developer contributors: 

        % table of stats for subset big10 dataset N=10: number of devs; average commits; average repo age; average number of issue tickets / PRs; 

        % PLOT: issue ticket and PR usage yes vs no
        % UPGRADE: split yes into assigned vs unassigned 
        
        \begin{tikzfigure}[This is the epcc logo]
            \includegraphics[height=100mm]{epcclogo.png}
        \end{tikzfigure}
        Quisque varius id dolor sed congue. Proin mollis lorem et tellus sodales, sit amet hendrerit urna vulputate. Cras maximus, dolor non laoreet vulputate, nisl nisl cursus ex.} 
        }
\block[linewidth=3pt]{}{
    \textbf{Acknowledgements:} Funded by EPSRC funding details details details; with abundant gratitude to my supervisors and love to my cats.}

\column{.5}
\block{3: Results}{
        \begin{tikzfigure}[]
        %\setlength{\belowcaptionskip}{-8pt}
            \includegraphics[width=0.45\linewidth]{epcclogo.png}
            \includegraphics[width=0.45\linewidth]{epcclogo.png}
        \end{tikzfigure}
    \begin{multicols}{2}
    
    {\fontsize{40}{40}\selectfont 
    \textbf{Context \& Demographics} \newline 
    
    TEXT ABOUT CONTENT AND DEMOGRAPHICS ETC ETC ETC 
    % RS repos with github.coms vs not. 
    % repo age 
    % repo size (N commits) 
    % dev team size (N Devs) 
    % repo recent activity  
    % languages 
    % Repos with issues and prs vs Not using.
    % numbers of issues and PRs
    \vspace*{0.4em} 
    \par
    
    \textbf{Assignment Practices} \newline  
    NNN of NNN items in NNN repositories were assigned to 1+ developers.   
    % BEWARE! _s need to be escaped for usage. 
    %\vspace*{0.1em}
    % TABLE OF ASSIGNMENT CATEGORIES PERCENTAGE
    \begin{center}
    \begin{tabular}{ l c c c} 
     %\hline
     \textbf{Assignment}\\ \textbf{Category} & \textbf{Devs (\%)} \\
     %\hline
     BOTH & 00 \\ 
     %\hline
     ISSUES\_ONLY & 00 \\
     %\hline
     PRs\_ONLY & 00 \\
     %\hline
     NEITHER & 00 \\ 
     %\hline
    \end{tabular}
    \end{center}
    %\vspace*{0.1em}
    Assignment types were categorised as \textit{being assigned to 1+ Issue and/or 1+ PR}
    \vspace*{0.4em} 
    \par
    
   
    \textbf{Commit Contributions} \newline 
    Nunc vitae tempor metus, sed varius tortor. Donec dapibus accumsan tempor. Praesent a augue accumsan, consectetur libero quis, suscipit nunc. In eu tortor mi. In dolor augue, tempus eget velit nec, ornare fermentum nibh. Pellentesque sollicitudin faucibus pharetra.   
    \vspace*{0.4em} 
    \par
    
    \textbf{Developer Interaction Types} \newline 
    Suspendisse porta, metus vel eleifend molestie, tortor justo pulvinar nulla, vel volutpat nulla nulla vitae nulla. Donec tristique lacinia rutrum.  Aenean dui mi, aliquam nec elit egestas.
    \par
    }
    \end{multicols}
}
\block{4: Discussion \& Evaluation}{
    {\fontsize{40}{40}\selectfont 
    \textbf{Findings} \newline Suspendisse eget diam sed ipsum pellentesque venenatis. Quisque ut convallis nisi, nec tempus erat. Praesent posuere odio in fringilla fringilla. Quisque ut dolor vitae urna pretium consequat. Cras nec magna nec ipsum semper maximus, per conubia nostra, per inceptos himenaeos. \newline
    \vspace*{0.4em} \newline
    \textbf{Validity} \newline Phasellus tempus eu tortor elementum imperdiet. Donec dignissim sodales purus. \newline
    }
}
\block[linewidth=3pt]{}{
    \textbf{References:} Lorem ipsum dolor sit amet, consectetur adipiscing elit. Morbi condimentum ipsum tortor, sit amet elementum eros auctor sed.}
\end{columns}
\end{document}